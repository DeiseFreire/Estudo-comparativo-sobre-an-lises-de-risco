\documentclass[a4paper,12pt]{report}
\usepackage[utf8]{inputenc} % Codificação de caracteres UTF-8
\usepackage[brazil]{babel} % Idioma português (com acentuação e formatação)
\usepackage[T1]{fontenc} % Codificação de fontes
%\usepackage{question} 
\usepackage{hyperref} % Links automáticos e numeração de páginas
\usepackage{tocloft} % Formatação de títulos e seções
%\usepackage{todonotes}
%\usepackage{enumitem}

\title{Estudo Digirido} % Título do documento
\author{Deise Freire} % Nome do autor
\date{\today} % Data atual

\begin{document}

\maketitle % Exibe o cabeçalho com título, autor e data

\tableofcontents % Gera um índice


%\begin{document}

\section{Introdução}

\subsection{Qual a importância da análise de risco de segurança da informação?}

A análise de risco é fundamental para proteger a informação e garantir o bom funcionamento dos processos de negócio.

\subsection{Quantos métodos de análise de risco existem?}

Existem muitos métodos de análise de risco disponíveis, cada um com suas características e finalidades específicas.

\section{Metodologia}

\subsection{Quais métodos de análise de risco foram selecionados?}

Os métodos CORAS, CIRA, ISRAM e IS foram selecionados por serem bem documentados e acessíveis online.

\subsection{Como os métodos foram comparados?}

Os métodos foram comparados usando o esquema de classificação de Campbell et al.

\section{Resultados}

\subsection{Quais as principais diferenças entre os métodos?}

Os métodos diferem em termos de propósito, entrada, resultado, esforço, escalabilidade, metodologia etc.

\subsection{Qual o melhor método de análise de risco?}

Não existe um método único que seja o melhor para todas as situações. A escolha do método depende das necessidades específicas da organização.

\section{Conclusão}

\subsection{Quais as implicações deste estudo para a prática?}

Este estudo pode ajudar as organizações a escolher o método de análise de risco mais adequado para suas necessidades.

\subsection{Quais as direções futuras de pesquisa?}

Estudos futuros podem comparar outros métodos de análise de risco ou investigar como os métodos podem ser melhorados.

\section{Perguntas específicas sobre os métodos}

\subsection{Qual o propósito do método CORAS?}

O CORAS visa identificar e avaliar os riscos de segurança da informação em uma organização.

\subsection{Quais as entradas do método CIRA?}

As entradas do CIRA incluem ativos de informação, ameaças e vulnerabilidades.

\subsection{Qual o resultado do método ISRAM?}

O ISRAM gera uma lista de riscos de segurança da informação classificados por sua importância.

\subsection{Como o método IS é diferente dos outros métodos?}

O método IS é baseado em um modelo de negócios e leva em consideração o contexto da organização.

\subsection*{Referência Bibliográfica}

%\bibitem{Agrawal2015}
Vivek Agrawal, "A Comparative Study on Information Security Risk Analysis Methods," The Norwegian Information Security Journal, vol. None, no. None, pp. None, Oct. 2015.

%\end{Referência bibliográfica}

\end{document}

